\documentclass{standalone}

\usepackage{castle_of_magic}

\begin{document}
\begin{tikzpicture}
\pic () at (0,0) {cardbackdisplay={Arcana}};
\end{tikzpicture}
\end{document}